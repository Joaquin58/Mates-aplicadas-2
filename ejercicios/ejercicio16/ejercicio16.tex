\chapter*{STEWART SECCION 3.7 EJERCICIO 16}
El volumen de una célula esférica en crecimiento es $V=\frac{4}{3}\pi r^3$, donde el radio $r$ se mide en micrómetros ($1\;\mu m = 10^{-6}m$).
\begin{enumerate}[label=\alph*)]
	\item Encuentre la razón de combio promedio de $V$ respecto a $r$, cuando este cambia de
	      \begin{multicols}{6}
		      \noindent
		      \begin{enumerate}[label=\roman*)]
			      \item 5 a 8 $\mu m$
			            \columnbreak
			      \item 5 a 6 $\mu m$
			            \columnbreak
			      \item 5 a 5.1 $\mu m$
		      \end{enumerate}
	      \end{multicols}
	\item Halle la razón de cambio instantánea de $V$ respecto a $r$, cuando $r=5\mu m$
	\item Demuestr que la razónde cambio del volumen de una esfera respecto a su radio es igual a su área superficial.
	      Explique geométricamente por qué esto es cierto. Arguemnte por analogía con el ejercicio 13c)
\end{enumerate}

\textbf{Inciso a)} Con base en que la razón de cambio de $y$ respecto de $x$ en el intervalo $[x_1,  x_2]$ puede interpretarse como la pendiente de la recta secante, esot es el cociente de diferencias $$\frac{\Delta y}{\Delta x}=\frac{f(x_2)-f(x_1)}{x_2-x_1}$$
Para este caso, nuestra función del Volumen que depende del radio $$V(r)=\frac{4}{3}\pi r^3$$. Para saber la razón de cambio promedio es necesario efectuar el razonamiento anterior, dado:$$\frac{\Delta V}{\Delta r}=\frac{V(r_1)-V(r_2)}{r_1-r_2}$$
para
\begin{multicols}{6}
	\noindent
	\begin{enumerate}[label=\roman*)]
		\item $r_1=5$ a $r_2=8$ $\mu m$
		      \columnbreak
		\item $r_1=5$ a $r_2=6$ $\mu m$
		      \columnbreak
		\item $r_1=5$ a $r_2=5.1$ $\mu m$
	\end{enumerate}
\end{multicols}
$\therefore \text{Sustituyendo en la ecuación}$
\begin{multicols*}{3}
	\noindent
	\begin{align*}
		\implies I)\; \frac{\Delta V}{\Delta r} = & \frac{V(8)-V(5)}{8-5}                               \\
		\implies\frac{\Delta V}{\Delta r} =       & \frac{\frac{4}{3}\pi(8)^3-\frac{4}{3}\pi(5)^3}{8-5} \\
        =       & \frac{\frac{4}{3}\pi(8^3-5^3)}{3}\\
        =       &\frac{\frac{4}{3}\pi(512-125)}{3}\\
        =       &\frac{\frac{4}{3}\pi(387)}{3}\\
        =       &\frac{4}{9}\pi(387)\\
		\simeq                                    & 540.3539 \; \mu m^3/\mu m
	\end{align*}
	\columnbreak
	\begin{align*}
		\implies II)\; \frac{\Delta V}{\Delta r} = & \frac{V(6)-V(5)}{6-5}                               \\
		\implies\frac{\Delta V}{\Delta r} =        & \frac{\frac{4}{3}\pi(6)^3-\frac{4}{3}\pi(5)^3}{6-5} \\
		=                                          & \frac{\frac{4}{3}\pi((6)^3-(5)^3)}{1}               \\
		=                                          & \frac{4}{3}\pi((6)^3-(5)^3)                         \\
		=                                          & \frac{4}{3}\pi(91)                                  \\
		\simeq                                     & 381.1800\; \mu m^3/\mu m                                    \\
	\end{align*}
	\columnbreak
	\begin{align*}
		\implies III)\; \frac{\Delta V}{\Delta r} = & \frac{V(5.1)-V(5)}{5.1-5}                               \\
		\implies \frac{\Delta V}{\Delta r} =        & \frac{\frac{4}{3}\pi(5.1)^3-\frac{4}{3}\pi(5)^3}{5.1-5} \\
		=                                           & \frac{\frac{4}{3}\pi((5.1)^3-(5)^3)}{0.1}               \\
		=                                           & \frac{4\pi(7.651)}{3\cdot 0.1}                          \\
		=                                           & \frac{4}{0.3}\cdot \pi(7.651)                           \\
		\simeq                                      & 320.4843\;\mu m^3/\mu m
	\end{align*}
\end{multicols*}
$\therefore$ Las razones medias de coambio para los valores de I), II) y III) respectivamente son
$$I)\simeq 810.5310 \; \mu m$$
$$II)\simeq 381.1800\; \mu m$$
$$III)\simeq 320.4843\;\mu m$$
\textbf{Inciso b)} Su límite, cuando $\Delta x \rightarrow 0$ es la derivada $f'(x_1)$, la cual puede interpretarse como la razón de cambio instantánea de y respecto a x, o sea, la pendiente de la recta tangente en $P(x_1, f(x_1))$.
Dicho esto la derivada de $V(r)$ es la razón de coambio instantánea: $$\therefore \left.\frac{dV}{dr}\right|_{r=5\;\mu m}=\left . 4\pi r^2\right|_{r=5\;\mu m}$$
$$\implies 4\pi (5)^2\simeq 314.1592 \;\mu m^3/\mu m$$
