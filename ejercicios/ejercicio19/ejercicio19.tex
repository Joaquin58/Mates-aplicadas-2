\chapter*{STEWART SECCION 3.7 EJERCICIO 19}

\textbf{19.} La cantidad de carga, $Q$, en coulombs $c)$ que ha pasado por un punto de un alambre hasta el tiempo $t$ (medidio en segundos) se expresa con $Q(t) = t^{3} - 2t_{2} + 6t +2$. Encuentre la corriente cuando $ a) t = 0.5s$ y $b) t = 1s$ [Véase el ejemplo 3. La unidad de corriente es el ampere ($1A = 1C/s$).] ¿En qué momento la corriente es la más baja?\\
\newline
\textbf{Desarrollo:}\newline
De acuerdo al ejemplo 3, la corriente es la rapidez con que la carga fluye por una suerficie. (C/s) = A.\\
\newline
$\Delta Q$ es la carga y $\Delta t$ es el periodo de tiempo. La corriente promedio en un intervalo de tiempo se da por $\frac{\Delta Q}{\Delta t}$, tomando el límite de la corriente en lapsos de tiempo muy pequeños $\Delta t \rightarrow 0$, obtenemos a lo que llamamos la corriente $I$ en un instante dado $t_{1}$.\\
\newline
\[
I = \lim_{\Delta t \to 0 }{\frac{\Delta Q}{\Delta t}} = \frac{dQ}{dt}
\] \\
la corriente es la derivada de la carga.\\
\newline
\textbf{Carga en Función del tiempo:}
\[
Q(t) = t^{3} - 2t^{2} + 6t + 2 
\]
\textbf{Derivar: }Derivando $Q(t)$ para obtener función de corriente.
\begin{align*}
    \frac{d}{dt}Q(t) &= \frac{d}{dt}(t^{3}-2t^{2}+6t+2)\\
    \frac{dQ}{dt} &= \frac{d}{dt}t^{3} - \frac{d}{dt}2t^{2} + \frac{d}{dt}6t + \frac{d}{dt}2\\
    \frac{dQ}{dt} &= 3t^{2} - 4t +6\\
    I &= 3t^{2} - 4t +6\\
\end{align*}
\textbf{Evaluando en los tiempos dados:}\\
\newline
\textbf{a)} $t = 0.5 s$
\begin{align*}
    I(0.5) &= 3(0.5)^{2} - 4(0.5) +6\\
    I(0.5) &= 3(0.25) - 4(0.5) +6\\
    I(0.5) &= 0.75 - 2 +6\\
    I(0.5) &= 4.75 A\\
\end{align*}
\textbf{b)} $t = 1 s$
\begin{align*}
    I(1) &= 3(1)^{2} - 4(1) +6\\
    I(1) &= 3(1) - 4 +6\\
    I(1) &= 3 - 4 +6\\
    I(1) &= 5 A\\
\end{align*}
\textbf{¿En qué momento la corriente es la más baja?}
\[
I(t_{min}) = I_{min}
\]
Primero, derivar la función $I(t)$.
\begin{align*}
    \frac{d}{dt}I(t) &= \frac{d}{dt}(3t^{2} -  +6)\\
    \frac{dI}{dt} &= \frac{d}{dt}3t^{2} - \frac{d}{dt}4t + \frac{d}{dt}6\\
    \frac{dI}{dt} &= 6t -4\\
\end{align*}
Resolver $\frac{dI}{dt} = 0$
\begin{align*}
    \frac{d}{dt}I(t) &= 0\\
    6t -4 &= 0\\
    6t &= 4\\
    t &= \frac{4}{6}\\
    t &= \frac{2}{3}\\
    t_{min} &= \frac{2}{3}\\
\end{align*}
Dado que en este punto la corriente presenta un mínimo es la más baja.