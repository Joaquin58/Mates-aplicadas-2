\chapter*{CONSTRUCCIÓN DE UNA MONTAÑA RUSA}

Suponga que se le solicita diseñar el primer ascenso y descenso de un nuevo modelo de cohete. Después de estudiar fotografías de sus momentos rusos y precedentes, decide hacer la pendiente de ascenso 0.8 y la de descenso -1.6. Opta por conectar estos dos tramos rectos y es \( L(x) \) en pies para el tramo que parte del suelo y es \( f(x)= ax^2 + bx + c \), donde \( f'(x) \), su derivada, es \( 2ax + b \). Mediante el trabajo ya mencionado, puede calcular ambos coeficientes de dirección por lo que dispone que los segmentos para \( L \) y \( S \), sean tangentes al parabolar en los puntos P y Q.

\begin{enumerate}[label=\alph*)]
    \item Suponga que la distancia horizontal entre P y Q es 100 pies. Escriba ecuaciones en \( a \), \( b \) y \( c \).
    Que hagan que la derivada de \( f \) sea 0.8 para el punto bajo estudio.
    Que resuelvan las trayectorias del inicio al suave en las puntas una fórmula para \( f(x) \).
    \item Dibuje \( L \) y \( S \), la parábola, y verifique gráficamente que las transiciones sean suaves.
    \item Encuentre la diferencia en elevación entre P y Q.
\end{enumerate}
    